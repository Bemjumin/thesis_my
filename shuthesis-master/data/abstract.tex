% 中英文摘要和关键字
\begin{cabstract}
    随着大模型的基础理论研究日趋成熟,其优秀的多模态理解能力和生成能力被应用于众多技术框架和复杂场景工业辅助设计、多参数控制系统运维决策等垂直领域中。然而在大模型的实际应用场景中仍面临许多困难和挑战。大模型本身的优越性能依赖于其庞大的参数及计算资源,并且在实际使用的推理框架中还涉及文本嵌入、知识检索等其他功能模块,这些降低了大模型推理框架的时效性。同时大模型在数据分析领域的应用也有一定的局限,尤其是在表格数据推理方面,大模型对表格数据的推理能力有限,无法直接处理复杂的表格数据,从而造成明显的幻觉问题。面对以上问题,本文针对大模型在检索增强生成( Retrieval-Augmented Generation ,RAG)技术框架的应用展开研究,重点研究模型轻量化压缩、表格数据推理优化等关键问题,提高大模型推理效率和对于表格数据的推理能力。并在水利水库调度领域开拓了应用场景,设计并实现了该领域下的基于本地知识库的大模型智能问答系统。本文的主要工作归纳如下: 

    1、为提高RAG系统的时效性,针对一种RAG系统中常见的连续提示词编码网络——循环神经网络,提出一种基于本征正交分解(Proper Orthogonal Decomposition, POD)的动力学系统模型压缩方法,通过循环神经网络隐藏状态的低维投影,将模型参数量与计算量分别压缩至原始模型的34.4\%与25.6\%,精度损失控制在5\%以内,并且通过将神经网络结构表达为动力学系统微分方程的形式,提高了其可解释性。
    
    2、为提高RAG系统的表格数据推理能力,针对表格数据特点及大模型对结构化表格数据推理的局限性,设计多级表格检索和推理框架,结合Agent代理思想及SQL代码生成技术,利用数据库技术简化表格,并构建检索-优化-生成的循环自纠错机制,优化了推理质量。
    
    3、针对数字孪生灌区目前存在的智能问答系统缺失,无法适应调度系统升级需求等问题,设计并实现了数字孪生灌区多功能智能知识库,结合RAG技术为智能问答系统提供本地知识支撑。通过建立业务规则库来支撑业务场景的规则适配,规范和约束水利业务管理行为。

    % 4、应用上述基于本征正交分解的动力学系统模型压缩方法、多级表格检索和推理框架等方法在实际场景中的优化效果,设计并实现了基于知识库的RAG智能问答系统。将检索增强生成技术与本地知识库技术结合并成功应用于水库调度、灌溉管理等场景,提升了水利水库调度领域的智能化水平。
    4、为研究RAG系统在实际垂直场景中的应用技术,设计并实现了在水利调度场景下的基于知识库的RAG智能问答系统并在问答系统中利用了本文提出的表格数据推理框架来提升推理效果。将检索增强生成技术与本地知识库技术结合并成功应用于水库调度、灌溉管理等场景,提升了水利水库调度领域的智能化水平。
    \end{cabstract}
    
    \ckeywords{大模型,模型压缩,表格数据推理,智能问答系统,检索增强推理}
    
    \begin{eabstract}
        As the theoretical foundations of large language models (LLMs) become increasingly mature, their exceptional capabilities in multi-modal understanding and generation have been applied to various technical frameworks and complex industrial scenarios, including industrial-assisted design and multi-parameter control system operations. However, practical applications of these models still face numerous challenges. The superior performance of large language models relies heavily on their vast parameter sizes and computational resources. Additionally, the reasoning frameworks of these models involve auxiliary modules such as text embedding and knowledge retrieval, which can reduce the timeliness of the reasoning process. Furthermore, large language models exhibit limitations in data analysis, particularly in tabular data reasoning, where their ability to handle complex tabular data is restricted, leading to significant hallucination issues. 

        To address these challenges, this study focuses on the application of Retrieval-Augmented Generation (RAG) technology frameworks, with an emphasis on model lightweight compression and optimization of tabular data reasoning capabilities. The goal is to enhance the reasoning efficiency of large language models and improve their reasoning capabilities for tabular data. Furthermore, this study explores the application of these technologies in the field of reservoir scheduling for water conservancy, designing and implementing an intelligent Q\&A system based on a local knowledge base for this domain. The main contributions of this work are summarized as follows:

        1. To improve the timeliness of RAG systems, this study proposes a model compression method based on Proper Orthogonal Decomposition (POD) for a common continuous prompt encoding network in RAG systems—recurrent neural networks. By projecting the hidden states of recurrent neural networks into a low-dimensional space, the model parameters and computational requirements are reduced to 34.4\% and 25.6\% of the original model, respectively, with a precision loss controlled within 5\%. Additionally, by expressing the neural network structure as a dynamic system differential equation, the interpretability of the model is enhanced.

        2. To enhance the tabular data reasoning capabilities of RAG systems, this study designs a multi-level tabular retrieval and reasoning framework. By integrating agent-based ideas and SQL code generation technology, this framework leverages database techniques to simplify tabular data and constructs a cyclic self-correction mechanism for retrieval, optimization, and generation, thereby improving reasoning quality.

        3. To address the lack of intelligent Q\&A systems in digital twin irrigation districts and their inability to meet the demands of upgraded scheduling systems, this study designs and implements a multi-functional intelligent knowledge base for digital twin irrigation districts. By integrating RAG technology, this knowledge base provides local knowledge support for the Q\&A system. Furthermore, a business rule library is established to support rule adaptation in business scenarios and to standardize and constrain water conservancy business management practices.

        % 4. By applying the proposed POD-based dynamic system model compression method and multi-level tabular retrieval and reasoning framework, this study designs and implements a knowledge base-based RAG intelligent Q\&A system. The integration of retrieval-augmented generation technology with local knowledge base technology has been successfully applied to scenarios such as reservoir scheduling and irrigation management, thereby enhancing the level of intelligence in the field of water conservancy reservoir scheduling.
        4. To explore the application technology of RAG systems in practical vertical fields, a knowledge-based RAG intelligent Q\&A system for water conservancy scheduling scenarios is designed and implemented. The tabular data reasoning framework proposed in this thesis is integrated into the Q\&A system to improve its reasoning effectiveness. Furthermore, by combining retrieval-augmented generation technology with local knowledge base technology, this approach has been successfully applied to reservoir scheduling, irrigation management, and other scenarios, thereby enhancing the intelligence level of reservoir scheduling in water conservancy.
    \end{eabstract}
    
    \ekeywords{ Large Language Models, Model Compression, Table Data Reasoning, Intelligent Question Answering System, RAG}
    