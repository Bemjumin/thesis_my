% !Mode:: "TeX:UTF-8"
\chapter{问题集样例}
\label{cha:问题集样例}
% 论文中用到的经典不等式.\\

% \noindent{\bfseries (H\"older Inequality)}
% 设~$a_i\geq0$, $b_i\geq0$, $i=1$, $2$, $\cdots$, $n$, 且~$p>1$, $q>1$ 
% 满足~$1/p+1/q=1$. 则有
% \[
% \sum_{i=1}^{n}a_ib_i\leq\left(\sum_{i=1}^{n}a_i^p\right)^{\frac1p}
% \cdot\left(\sum_{i=1}^{n}b_i^q\right)^{\frac1q},
% \]
% 等号成立当且仅当存在一个常数~$c$ 满足~$a_i^p=cb_i^q$.\\

% \noindent{\bfseries (PM Inequality)}
% 设~$x_1$, $x_2$, $\ldots$, $x_n$ 是~$n$ 个非负实数. 如果~$0<p<q$, 那么
% \[
% \left(\frac{x_1^p+x_2^p+\cdots+x_n^p}{n}\right)^{\frac{1}{p}}\leq
% \left(\frac{x_1^q+x_2^q+\cdots+x_n^q}{n}\right)^{\frac{1}{q}},
% \]
% 等号成立当且仅当~$x_1=x_2=\cdots =x_n$.\\

% \noindent{\bfseries (AM-GM Inequality)}
% 设~$x_1$, $x_2$, $\ldots$, $x_n$ 是~$n$ 个非负实数. 则有
% \[
% \frac{x_1+x_2+\cdots+x_n}{n}\geq\sqrt[n]{x_1x_2\cdots x_n},
% \]
% 等号成立当且仅当~$x_1=x_2=\cdots =x_n$.
样例1:

表名: 
三里店洪水汛期调度运用计划表

问题:
三里店水库百年一遇洪水的入库流量是多少?

表格数据:

\begin{table}[htbp]
    \centering
    % \caption{三里店水库汛期调度运用计划表}
    % \label{tab:flood-plan}
    \begin{tabular}{llllc}
    \toprule
    \textbf{洪水标准} & \textbf{入库流量} & \textbf{入库洪水总量} & \textbf{迎洪控制水位} & \textbf{泄流措施} \\
    \midrule
    5 & 43.3 & 42 & 汛限水位 & 开启输水建筑物水塔闸门 \\
    10 & * & * & 汛限水位 & 开启输水建筑物水塔闸门 \\
    20 & * & * & 汛限水位 & 开启输水建筑物水塔闸门 \\
    30 & * & * & 汛限水位 & 开启输水建筑物水塔闸门 \\
    50 & * & * & 汛限水位 & 开启输水建筑物水塔闸门 \\
    100 & * & * & 汛限水位 & 开启输水建筑物水塔闸门 \\
    500 & * & * & 汛限水位 & \makecell{开启输水建筑物水塔闸门,\\同时开启溢洪道泄洪} \\
    \bottomrule
    \end{tabular}
\end{table}
\newpage
样例2:

表名:隆德县农业灌溉用水权确权汇总表

问题:列出渝河库罐区所有确权水量大于20万立方米的村

表格数据(部分):

\begin{table}[htbp]
    \centering
    \footnotesize
    % \caption{隆德县农业灌溉用水权确权汇总表}
    % \label{tab:irrigation-rights}
    % \begin{tabular}{p{1.5cm}p{1.5cm}p{1.5cm}p{1.5cm}p{1.5cm}p{1.5cm}p{1.5cm}}
    \begin{tabular}{lllllll}
    \toprule
    \textbf{乡(镇)} & \textbf{灌区名称} & \textbf{灌溉方式} & \textbf{确权面积} & \textbf{净灌溉定额} & \textbf{\makecell{确权单元灌溉\\水利用系数}} & \textbf{确权水量} \\
    \midrule
    峰台社区 & 渝河库井灌区 & 高效节灌 & 360.9471 & 115 & 0.883 & 18.19201 \\
    竹林村 & 渝河库井灌区 & 高效节灌 & * & 115 & 0.883 & 10.96251 \\
    杨店村 & 渝河库井灌区 & 高效节灌 & * & 115 & 0.883 & 4.688562 \\
    咀头村 & 渝河库井灌区 & 高效节灌 & * & 115 & 0.883 & 2.344281 \\
    吴山村 & 渝河库井灌区 & 高效节灌 & * & 115 & 0.883 & 1.953567 \\
    三合村 & 渝河库井灌区 & 高效节灌 & * & 115 & 0.883 & 4.455436 \\
    红崖社区 & 渝河库井灌区 & 高效节灌 & * & 115 & 0.883 & 7.185221 \\
    南河村 & 渝河库井灌区 & 高效节灌 & * & 115 & 0.883 & 2.14111 \\
    星火社区 & 渝河库井灌区 & 高效节灌 & * & 115 & 0.883 & 3.296971 \\
    新兴村 & 渝河库井灌区 & 高效节灌 & * & 115 & 0.883 & 12.28143 \\
    陈靳村 & 渝河库井灌区 & 高效节灌 & * & 115 & 0.883 & 7.605889 \\
    光联村 & 渝河库井灌区 & 高效节灌 & * & 115 & 0.883 & 15.4675 \\
    和平村 & 渝河库井灌区 & 高效节灌 & * & 115 & 0.883 & 26.00687 \\
    马河村 & 渝河库井灌区 & 高效节灌 & * & 115 & 0.883 & 10.84632 \\
    清泉村 & 渝河库井灌区 & 高效节灌 & * & 115 & 0.883 & 27.56523 \\
    街道村 & 渝河库井灌区 & 高效节灌 & * & 115 & 0.883 & 44.81894 \\
    十八里村 & 渝河库井灌区 & 高效节灌 & * & 115 & 0.883 & 36.89222 \\
    新民村 & 渝河库井灌区 & 高效节灌 & * & 115 & 0.883 & 43.39429 \\
    许沟村 & 渝河库井灌区 & 高效节灌 & * & 115 & 0.883 & 41.2671 \\
    张树村 & 渝河库井灌区 & 高效节灌 & * & 115 & 0.883 & 45.08452 \\
    锦华村 & 渝河库井灌区 & 高效节灌 & * & 115 & 0.883 & 9.384278 \\
    锦屏村 & 渝河库井灌区 & 高效节灌 & * & 115 & 0.883 & 4.700896 \\
    \bottomrule
    \end{tabular}
\end{table}
\newpage
样例3:

表名:打食沟水库洪水汛期调度运用计划表

问题:打石沟水库面对5年一遇洪水时的泄洪措施是什么

表格数据:

\begin{table}[htbp]
    \centering
    \footnotesize
    % \caption{打食沟水库洪水汛期调度运用计划表}
    % \label{tab:flood-plan}
    \begin{tabular}{lllll}
    \toprule
    \textbf{洪水标准(年)} & \textbf{库流量} & \textbf{洪水总量} & \textbf{迎洪控制水位} & \textbf{泄流措施} \\
    \midrule
    5 & 19.2 & 12.2 & 汛限水位 & 开启输水建筑物水塔闸门 \\
    10 & 26.8 & 17 & 汛限水位 & 开启输水建筑物水塔闸门 \\
    20 & 37.4 & 23.7 & 汛限水位 & 开启输水建筑物水塔闸门 \\
    30 & 44.1 & 27.9 & 汛限水位 & 开启输水建筑物水塔闸门 \\
    50 & 53.2 & 33.7 & 汛限水位 & 开启输水建筑物水塔闸门 \\
    100 & 66 & 41.8 & 汛限水位 & 开启输水建筑物水塔闸门 \\
    300 & 86.7 & 54.9 & 汛限水位 & \makecell{开启输水建筑物水塔闸门\\,同时开启溢洪道泄洪} \\
    \bottomrule
    \end{tabular}
    \end{table}